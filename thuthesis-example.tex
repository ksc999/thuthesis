% !TeX encoding = UTF-8
% !TeX program = xelatex
% !TeX spellcheck = en_US

\documentclass[degree=master]{thuthesis}
  % 学位 degree:
  %   doctor | master | bachelor | postdoc
  % 学位类型 degree-type:
  %   academic(默认)| professional
  % 语言 language
  %   chinese(默认)| english
  % 字体库 fontset
  %   windows | mac | fandol | ubuntu
  % 建议终版使用 Windows 平台的字体编译


% 论文基本配置,加载宏包等全局配置
% !TeX root = ./thuthesis-example.tex

% 论文基本信息配置

\thusetup{
  %******************************
  % 注意:
  %   1. 配置里面不要出现空行
  %   2. 不需要的配置信息可以删除
  %   3. 建议先阅读文档中所有关于选项的说明
  %******************************
  %
  % 输出格式
  %   选择打印版(print)或用于提交的电子版(electronic),前者会插入空白页以便直接双面打印
  %
  output = print,
  %
  % 标题
  %   可使用“\\”命令手动控制换行
  %
  title  = {鲁棒控制障碍函数的验证与综合:多级多项式优化和半定松弛},
  title* = {Verification and Synthesis of Robust Control Barrier Function: Multilevel Polynomial Optimization and Semidefinite Relaxation},
  %
  % 学位
  %   1. 学术型
  %      - 中文
  %        需注明所属的学科门类,例如:
  %        哲学、经济学、法学、教育学、文学、历史学、理学、工学、农学、医学、
  %        军事学、管理学、艺术学
  %      - 英文
  %        博士:Doctor of Philosophy
  %        硕士:
  %          哲学、文学、历史学、法学、教育学、艺术学门类,公共管理学科
  %          填写“Master of Arts“,其它填写“Master of Science”
  %   2. 专业型
  %      直接填写专业学位的名称,例如:
  %      教育博士、工程硕士等
  %      Doctor of Education, Master of Engineering
  %   3. 本科生不需要填写
  %
  % degree-name  = {工学学士},
  % degree-name* = {Bachelor of Engineering},
  %
  % 培养单位
  %   填写所属院系的全名
  %
  department = {电子工程系},
  %
  % 学科
  %   1. 学术型学位
  %      获得一级学科授权的学科填写一级学科名称,其他填写二级学科名称
  %   2. 工程硕士
  %      工程领域名称
  %   3. 其他专业型学位
  %      不填写此项
  %   4. 本科生填写专业名称,第二学位论文需标注“(第二学位)”
  %
  discipline  = {电子信息科学与技术},
  discipline* = {Electronic Information Science and Technology},
  %
  % 姓名
  %
  author  = {康书诚},
  author* = {Kang Shucheng},
  %
  % 指导教师
  %   中文姓名和职称之间以英文逗号“,”分开,下同
  %
  supervisor  = {张旭东, 教授},
  supervisor* = {Professor Zhang Xudong},
  %
  % 副指导教师
  %
  % associate-supervisor  = {陈文光, 教授},
  % associate-supervisor* = {Professor Chen Wenguang},
  %
  % 联合指导教师
  %
  % co-supervisor  = {某某某, 教授},
  % co-supervisor* = {Professor Mou Moumou},
  %
  % 日期
  %   使用 ISO 格式;默认为当前时间
  %
  % date = {2019-07-07},
  %
  % 是否在中文封面后的空白页生成书脊(默认 false)
  %
  include-spine = false,
  %
  % 密级和年限
  %   秘密, 机密, 绝密
  %
  % secret-level = {秘密},
  % secret-year  = {10},
  %
  % 博士后专有部分
  %
  % clc                = {分类号},
  % udc                = {UDC},
  % id                 = {编号},
  % discipline-level-1 = {计算机科学与技术},  % 流动站(一级学科)名称
  % discipline-level-2 = {系统结构},          % 专业(二级学科)名称
  % start-date         = {2011-07-01},        % 研究工作起始时间
}

% 载入所需的宏包

% 定理类环境宏包
\usepackage{amsthm}
% 也可以使用 ntheorem
% \usepackage[amsmath,thmmarks,hyperref]{ntheorem}

\thusetup{
  %
  % 数学字体
  % math-style = GB,  % GB | ISO | TeX
  math-font  = xits,  % stix | xits | libertinus
}

% 可以使用 nomencl 生成符号和缩略语说明
% \usepackage{nomencl}
% \makenomenclature

% 表格加脚注
\usepackage{threeparttable}

% 表格中支持跨行
\usepackage{multirow}

% 固定宽度的表格。
% \usepackage{tabularx}

% 跨页表格
\usepackage{longtable}

% 算法
\usepackage{algorithm}
\usepackage{algorithmic}

% 量和单位
\usepackage{siunitx}

% 参考文献使用 BibTeX + natbib 宏包
% 顺序编码制
\usepackage[sort]{natbib}
\bibliographystyle{thuthesis-numeric}

% 著者-出版年制
% \usepackage{natbib}
% \bibliographystyle{thuthesis-author-year}

% 本科生参考文献的著录格式
% \usepackage[sort]{natbib}
% \bibliographystyle{thuthesis-bachelor}

% 参考文献使用 BibLaTeX 宏包
% \usepackage[style=thuthesis-numeric]{biblatex}
% \usepackage[style=thuthesis-author-year]{biblatex}
% \usepackage[style=apa]{biblatex}
% \usepackage[style=mla-new]{biblatex}
% 声明 BibLaTeX 的数据库
% \addbibresource{ref/refs.bib}

% 定义所有的图片文件在 figures 子目录下
\graphicspath{{figures/}}

% 数学命令
\makeatletter
\newcommand\dif{%  % 微分符号
  \mathop{}\!%
  \ifthu@math@style@TeX
    d%
  \else
    \mathrm{d}%
  \fi
}
\makeatother

% hyperref 宏包在最后调用
\usepackage{hyperref}



\begin{document}

% 封面
\maketitle

% 学位论文指导小组、公开评阅人和答辩委员会名单
% 本科生不需要
% \input{data/committee}

% 使用授权的说明
\copyrightpage
% 将签字扫描后授权文件 scan-copyright.pdf 替换原始页面
% \copyrightpage[file=scan-copyright.pdf]

\frontmatter
% !TeX root = ../thuthesis-example.tex

% 中英文摘要和关键字

\begin{abstract}
  本文研究了具备加性不确定性和凸控制输入边界的仿射多项式系统中,鲁棒控制障碍函数的验证和综合问题。在上述系统条件下,控制障碍函数的验证和综合问题可表示为多级多项式优化问题。其中,验证问题包含三个级别的优化:系统不确定性、控制输入和系统状态;而综合问题对控制障碍函数的候选参数也进行了优化。研究表明,通过对系统不确定性和控制输入这两个优化级别使用KKT条件,验证问题可以简化为单级多项式优化问题,综合问题可以简化为最小-最大多项式优化问题。进一步地,我们可以使用多级半正定松弛求解上述两个问题:对于验证问题转化而来的单级多项式优化问题,我们应用Lasserre's Hierarchy这一矩松弛方法;对于综合问题转化来的最小-最大多项式优化问题,我们使用平方和优化方法获得优化问题未知值函数的不断紧逼的多项式下界,并再次调用Lasserre's Hierarchy以最大化下界。两种松弛方法都拥有渐进收敛到全局最优的理论保证。实验方面,我们对受控Van der Pol振荡器进行了深入研究,并对系统具备和不具备加性不确定性这两种情况分别进行了讨论。

  % 关键词用“英文逗号”分隔,输出时会自动处理为正确的分隔符
  \thusetup{
    keywords = {控制障碍函数,多项式优化,半定松弛,最小-最大优化},
  }
\end{abstract}

\begin{abstract*}
  In this paper, we study the verification and synthesis of robust control barrier functions for affine polynomial systems with additive uncertainty and convex control input bounds. Under the above system conditions, the verification and synthesis problem of the control barrier function can be expressed as multilevel polynomial optimization problems. Among them, the verification problem contains three levels of optimization: system uncertainty, control input, and system state; while the synthesis problem also optimizes the candidate parameters of the control barrier function. It is shown that by using KKT conditions for the two optimization levels of system uncertainty and control input, the verification problem can be reduced to a single-level polynomial optimization problem, and the synthesis problem can be reduced to a min-max polynomial optimization problem. Furthermore, we can use multi-level positive semi-definite relaxation to solve the above two problems: for the single-level polynomial optimization problem transformed from the verification problem, we apply the Lasserre's Hierarchy moment relaxation method; for the min-max polynomial problem transformed from the synthesis problem, we use the sum-of-squares optimization method to obtain an ever-tightening polynomial lower bound for the unknown value function of the optimization problem, and call Lasserre's Hierarchy again to maximize the lower bound. Both relaxation methods have theoretical guarantees of asymptotic convergence to the global optimum. In terms of experiments, we have conducted an in-depth study on the controlled Van der Pol oscillator, and discussed the two cases of the system with and without additive uncertainty.

  % Use comma as separator when inputting
  \thusetup{
    keywords* = {control barrier function, polynomial optimization, semidefinite relaxation, min-max optimization},
  }
\end{abstract*}


% 目录
\tableofcontents

% 插图和附表清单
% 本科生的插图索引和表格索引需要移至正文之后、参考文献前
% \listoffiguresandtables  % 插图和附表清单(仅限研究生)
\listoffigures           % 插图清单
\listoftables            % 附表清单

% 符号对照表
\input{data/denotation}


% 正文部分
\mainmatter
\input{data/chap01}
\input{data/chap02}
\input{data/chap03}
\input{data/chap04}


% 其他部分
\backmatter

% 参考文献
\bibliography{ref/refs}  % 参考文献使用 BibTeX 编译
% \printbibliography       % 参考文献使用 BibLaTeX 编译

% 附录
% 本科生需要将附录放到声明之后,个人简历之前
\appendix
% \input{data/appendix-survey}       % 本科生:外文资料的调研阅读报告
% % !TeX root = ../thuthesis-example.tex

% \usepackage{amsmath,amsfonts}
% \usepackage{multicol}
% \usepackage{graphicx}

%definitions
\renewcommand{\AA}{{\cal{A}}}
\newcommand{\II}{{\cal{I}}}
\newcommand{\CC}{{\cal{C}}}
\newcommand{\FF}{{\cal{F}}}
\newcommand{\LL}{{\cal{L}}}
\newcommand{\R}{\mathbb{R}}
% \newcommand{\N}{\mathbb{N}}
\renewcommand{\SS}{{\cal{S}}}
\newcommand{\TT}{{\cal{T}}}
\newcommand{\UU}{{\mathbf{U}}}
\newcommand{\XX}{{\mathbf{X}}}
\newcommand{\I}{{\boldsymbol{1}}}
\def\z{\mathbf{z}}
\def\w{\mathbf{w}}
\def\y{\mathbf{y}}
\def\K{\mathbf{K}}
\newcommand{\comment}[1]{\textcolor{red}{\textbf{#1}}}

% \newtheorem{assumption}{Assumption}
% \newtheorem{example}{Example}
% \newtheorem{remark}{Remark}

\begin{translation}
\label{cha:translation}

\title{标题:通过占据测度实现非线性最优控制综合}
\author{Didier Henrion, Jean B. Lasserre and Carlo Savorgnan}
\maketitle

\begin{abstract}
  我们考虑所有问题数据都是多项式的非线性最优控制问题 (OCP)。 在本文的第一部分,我们回顾了如何使用线性矩阵不等式 (LMI) 松弛的层次结构,使用占用度量逐点逼近给定 OCP 的最优价值函数。 在第二部分中,我们将方法扩展到在给定集合上逼近最优值函数,并使用这样的函数来建设性和计算性地推导出几乎最优的控制律。数值例子表明了该方法的有效性。
\end{abstract}

\section{引言}

众所周知,尽管 Pontryagin 最小原理和 Hamilton-Jacobi-Bellman 最优条件等理论工具的强大功能,解决最优控制问题 (OCP) 可能是一项非常艰巨的任务。
在处理状态和输入约束时,这种说法尤其正确。


\textbf{贡献。} 在本文中,我们考虑所有问题数据都是多项式的 OCP 类。
我们部署的方法(在 \cite{LasPriHen2005} 中引入)基于矩理论,包括推导 OCP 的凸线性矩阵不等式 (LMI) 松弛的层次结构,它给出最优值下限的递增序列 . 这些 LMI 问题可以使用现成的半定规划 (SDP) 求解器来解决。

关于 \cite{LasPriHen2005} 及其扩展版本 \cite{LasHenPriTre2008} 的贡献是双重的。 首先,从基本概念开始,以更简单的方式获得松弛的推导。 第二个也是更重要的贡献是,我们展示了如何应用该方法来近似集合上的最优值函数,并以建设性和计算方式推导
一个控制律。 几个简单的例子说明了这种方法。

\textbf{符号。}
$\R$ 和 $\mathbb{N}$ 分别表示实数集和整数集。
$\R[y]=[y_1, \dots , y_n]$ 表示变量 $y$ 中的多项式环。
$\R[y]_d=[y_1, \dots , y_n]$ 表示变量 $y$ 中次数最多为 $d$ 的多项式环。
当 $y \in \R^n$ 和 $\alpha\in\mathbb{N}^n$ 时,$y^\alpha$ 代表 $y_1^{\alpha_1} \dots y_n^{\alpha_n}$。
给定一个多项式函数 $\varphi$,$\deg(\varphi)$ 是其单项式的最大次数。
给定一个可微函数 $\varphi(y)$, $\nabla_y(\varphi)=[\frac{\partial\varphi}{\partial y_1}, \dots ,\frac{\partial\varphi}{\partial y_n }]$ 是它相对于 $y$ 的梯度。
$\delta_{y_0}$ 是 $y_0$ 处的狄拉克测度。
$v'$ 表示 $v$ 的转置。

\section{问题定义}
考虑由微分方程描述的连续时间系统
\begin{equation}\label{eq:dynamics}
\dot x(t) = f(t, x(t), u(t))
\end{equation}
其中 $x\in\R^n$ 和 $u\in\R^m$ 分别是状态向量和输入向量。
通过定义代价函数
\begin{equation}\label{eq:cost}
\int_0^T h(t, x(t), u(t)) dt + H(x(T))
\end{equation}
初始约束为
\begin{equation}\nonumber
x(0) \in \CC_I=\{ x: g_{I_j}(x)\leq 0, ~ j=1,\dots,n_I \} 
\end{equation}
末态约束为
\begin{equation}\nonumber
x(T) \in \CC_F=\{ x: g_{F_j}(x)\leq 0, ~ j=1,\dots,n_F \}
\end{equation}
我们可以制定几个OCP。 例如,当 $\CC_I$ 和 $\CC_F$ 仅包含一个点时,我们遇到了将系统从指定的初始条件 $x(0)=x_0$ 驱动到最终条件 $x(T)=x_T$ 的经典问题 通过最小化给定的成本。

在续集中,我们将考虑在轨迹 $(t, x(t), u(t)) \in \CC_T$ 上附加约束的所有问题,其中
\begin{equation}\nonumber
\CC_T=\{ (t,x,u): g_{T_j}(t,x,u)\leq 0, ~ j=1,\dots,n_T \}.
\end{equation}

推导该方法所必需的一个重要假设是所有问题数据都是多项式的。 更确切地说:
\begin{assumption}\label{as:poly}
  函数 $f$、$h$、$H$、$g_{I_j}$、$g_{T_j}$ 和 $g_{F_j}$ 是多项式的。
\end{assumption}

\section{最优控制的矩方法}
矩方法的关键思想是定义三个 \textit{占用度量},它们传达有关系统初始条件、轨迹和最终条件的信息。
然后根据此类措施的时刻对 OCP 进行重新措辞。
得到的凸问题包含三个成分:
\begin{itemize}
\item 一组对表征系统动力学的矩的线性等式约束;
\item 一组半定约束,这些约束来自矩属于一个度量的事实;
\item 一组半定约束,它们将 $\CC_I$、$\CC_T$ 和 $\CC_F$ 引起的约束转换到测度的支持上。
\end{itemize}
为了得出这个约束,我们假设地平线 $T$ 是固定的。


\subsection{轨迹约束}
为了获得轨迹约束,我们从系统轨迹可以表征研究某些\textit{测试函数}如何沿着轨迹演变的想法开始。 为此,我们选择 $t^\alpha x^\beta$ 形式的单项式函数。 考虑轨迹 $x(t)$。 使用微积分基本定理我们可以写为
\begin{equation}\label{eq:ftoc}
T^\alpha x(T)^\beta = 0^\alpha x(0)^\beta + \int_0^T \frac{d(t^\alpha x(t)^\beta)}{dt} dt.
\end{equation}
轨迹约束是通过重新表述方程 (\ref{eq:ftoc}) 获得的
三个适当定义的占据测度。

\textit{最终占据测度} $\mu_F$ 捕获了时间 $T$ 的状态信息
\begin{equation}\nonumber
x(T)^\beta = \int x^\beta \delta_{x(T)}(dx) = \int x^\beta d\mu_F.
\end{equation}
\textit{初始占据测度} $\mu_I$ 捕捉系统初始状态的信息
\begin{equation}\nonumber
x(0)^\beta = \int x^\beta \delta_{x(0)}(dx) = \int x^\beta d\mu_I.
\end{equation}
\textit{轨迹占据测度} $\mu_T$ 捕获了关于 $t$, $x(t)$ 和 $u(t)$ 沿轨迹的值的信息
\[
\int_0^T \!\! t^\gamma x(t)^\eta u(t)^\nu dt \!
=\!\int_0^T \!\!\!\!\int t^\gamma x^\eta u^\nu \delta_{x(t),u(t)}(dx,du)dt \!
=\!\int x^\gamma u^\eta t^\nu d\mu_T.
\]
请注意 $\mu_F$ 和 $\mu_I$ 是概率度量,它们的质量等于 $1$。

接下来,如果 $f\in\R[t,x,u]$ 那么 $\frac{d(x^\alpha t^\beta)}{dt}\in\R[t,x,u]$ 因为
\begin{equation}\nonumber %\label{aux}
\frac{d(x^\alpha t^\beta)}{dt} \! = \! \frac{\partial(x^\alpha t^\beta)}{\partial t} + \nabla_x (x^\alpha t^\beta)f(x,u) \! = \!\! \sum_{\gamma,\eta,\nu} \! a^{\alpha\beta}_{\gamma\eta\nu} x^\gamma u^\eta t^\nu
\end{equation}
对于某些依赖于 $f$ 的系数 $a^{\alpha\beta}_{\gamma\eta\nu}$。
导数的次数是$\deg(x^\alpha t^\beta)-1+\deg(f)$。
使用前面的等式,(\ref{eq:ftoc}) 可以改写为
\begin{equation}\label{eq:meascons}
%= 0^\alpha \int x^\beta d\mu_I + \int \frac{(t^\alpha x^\beta)}{dt} d\mu_T \\
T^\alpha \int x^\beta d\mu_F 
= 0^\alpha \int x^\beta d\mu_I + \sum_{\gamma,\eta,\nu} a^{\alpha\beta}_{\gamma\eta\nu} \int t^\gamma x^\eta u^\nu d\mu_T,
\end{equation}
即,时刻之间的{\it 线性}关系
$\mu_F, \mu_I$ 和 $\mu_T$。
即,引入符号 
$z_{\beta}=\int x^\beta d\mu_F$,
$w_{\beta}=\int x^\beta d\mu_I$,
$y_{\gamma\eta\nu}=\int t^\gamma x^\eta u^\nu d\mu_T$, 
我们得到
\begin{equation}\label{eq:momcons}
T^\alpha z_{\beta} = 0^\alpha w_{\beta} + \sum_{\gamma,\eta,\nu} a^{\alpha\beta}_{\gamma\eta\nu} y_{\gamma\eta\nu}
\end{equation}
对于每个 $\alpha,\beta\in\mathbb{N}\times\mathbb{N}^{n}$。 请注意,从 (\ref{eq:momcons}),
$\mu_T$ 的质量是 $T$。
在紧凑的表示法中,考虑次数高达 $r$ 的测试函数和次数最多为 $r$ 的单项式的规范基础:
\begin{equation}\nonumber
m_r(x)=[1,x_1,\dots,x_n,x_1^2,x_1x_2,\dots,x_1^r,x_1^{r-1}x_2,\dots,x_n^r]'.
\end{equation}
定义向量
$\z_r=\int m_r(x) d\mu_F$,
$\w_r=\int m_r(x) d\mu_I$和
$\y_k=\int m_k(t,x,u) d\mu_T$.
Then,
\begin{equation}\label{eq:momcon}
A_F \z_r = A_I \w_r + A_T \y_k
\end{equation}
其中 $k\geq r-1+\deg(f)$ 和矩阵 $A_F$、$A_I$ 和 $A_T$ 的系数可以从等式 (\ref{eq:momcons}) 中获得。

将系数向量 $c_h$ 和 $c_H$ 定义为
\begin{equation}\nonumber
h(t,x,u)=c_h'm_k(x,t,u), \qquad H(x)=c_H' m_r(x).
\end{equation}
观察到
\begin{equation}\label{criterion}
\int_0^T h(t, x(t), u(t)) dt + H(x(T))=c_h' \y_k+  c_H' \z_r,
\end{equation}
即,OCP 的标准是 $\z_r$ 和 $\y_k$ 上的线性泛函。

到目前为止,对于给定的轨迹 $x(t)$,我们已经描述了三个相关占据测度的时刻满足的线性约束。 现在,如果轨迹未知,则三个度量未知,我们可以考虑抽象线性规划 (LP) 问题 $J(\mu_I) = \min_{\mu_T,\mu_I,\mu_F}
\int h d\mu_T + \int H d\mu_F$
受 (\ref{eq:meascons}) 影响,其目的是找到与最佳轨迹相关的占据测度。 $\mu_F$, $\mu_I$, $\mu_T$ 的特征是通过它们各自的截断矩向量 $\z_r$, $\w_r$, $\y_k$,
剩下的困难是找到条件,确保这些向量和测量的力矩向量分别支持 $\CC_F$、$\CC_I$、$\CC_T$。 这将在下一节中解释。

该方法的一个很好的特点是我们可以使用初始和最终措施。 例如,如果 $\mu_I=\delta_{x_0}$,我们检索具有固定初始状态 $x_0$ 的 OCP 的最优成本 $J(\delta_{x_0})$。 现在,如果 $\mu_I$ 未知,但已知支持 $\CC_I$,则 $J(\mu_I)=\min_{x_0\in\CC_I}J(\delta_{x_0})$。 最后,如果 $\mu_I$ 已知,但狄拉克未知,则求解上述 LP 问题旨在计算 $\int J(\delta_{x_0})d\mu_I(x_0)$。

\subsection{矩矩阵约束}
存在线性规划 (LP) 或半定规划 (SDP) 无限向量是 {\it moment} 向量的充分必要条件,即紧基本半代数集上某个有限 Borel 测度的矩向量 ; 参见例如 \cite{Put1993}。 我们选择了后者,因为它已被证明对数值目的更有效 \cite{LasPri2004}。

$r$ 为偶数,令
\begin{equation}\nonumber
M(\z_r)=\int m_{r/2}(x) m_{r/2}(x)' d\mu_F
\end{equation}
是与 $\mu_F$ 关联的 $r$ 阶 {\it 矩矩阵}。 显然,$M(\z_r)$是半正定的,记为$M(\z_r)\succeq0$。 因此,在 OCP 的凸弛豫中,施加
\begin{equation}\label{eq:mmc}
M(\z_r) \succeq 0,
\end{equation}
并且对 $\w_r$ 和 $\y_k$ 施加了类似的约束。

\subsection{定位矩阵约束}
与上一小节类似,可以根据 $\z_r、\w_r$ 和 $\y_k$ 上的线性矩阵不等式表示由 $\CC_I$、$\CC_T$ 和 $\CC_F$ 引起的支持约束。 为了推导出这样的不等式,定义
\begin{equation}\nonumber
d_{F_j}=\left\{
\begin{array}{ll}
\deg(g_{F_j}(x)) & \mbox{if}~\deg(g_{F_j}(x))~\mbox{is even} \\
\deg(g_{F_j}(x))+1 & \mbox{if}~\deg(g_{F_j}(x))~\mbox{is odd}
\end{array} \right.
\end{equation}
和定位矩阵
\begin{equation}\nonumber
L_{g_{F_j}}(\z_r)=\int g_{F_j}(x) m_{(r-d_{F_j})/2}(x) m_{(r-d_{F_j})/2}(x)' d\mu_F.
\end{equation}
矩阵 $g_{F_j}(x) m_{(r-d_{F_j})/2}(x) m_{(r-d_{F_j})/2}(x)'$ 对于每个值都是半正定的 $x$ 使得 $g_{F_j}(x)\geq 0$。 因此,如果 $\mu_F$ 在 $\CC_F$ 上得到支持,则 $L_{{F_j}}(\z_r)\succeq0$ 对于每个 $j$ 因此,在 OCP 的凸松弛中,施加半定约束
\begin{equation}\label{eq:lmc}
L_{g_{F_j}}(\z_r) \succeq 0 \qquad j=1,\dots,n_F
\end{equation}
以及 $\w_r$ 和 $\y_k$ 的类似半定约束。

有关力矩和局部化矩阵约束的更多详细信息,请参见 \cite{Las2001}。

\subsection{凸松弛}
为了构造 OCP 的凸松弛,设 $r$ 和 $k$ 为偶数,使得
\begin{equation}\nonumber
r\geq\deg(H), \quad k\geq\deg(h), \quad k\geq r+\deg(f).
\end{equation}
在本文中,我们将假设初始概率
测量 $\mu_I$ 通过其矩 $\w_r$ 已知。

凸松弛是以下截断矩问题:
\begin{equation}\label{eq:momocp2}
\begin{split}
\min_{\z_r, \y_k}\quad & c_h' \y_k+  c_H' \z_r \\
& A_F \z_r = A_I \bar \w_r + A_T \y_k \\
& M(\z_r)\succeq 0, ~~ L_{g_{F_j}}(\z_r)\succeq 0, ~~ \forall j=1,\dots,n_F \\
& M(\y_k)\succeq 0, ~~ L_{g_{T_j}}(\y_k)\succeq 0, ~~ \forall j=1,\dots,n_T
\end{split}
\end{equation}
其中符号 $\bar \w_r$ 表示力矩向量是已知的。
当下问题应注意两个重要事实 (\ref{eq:momocp2}):
\begin{itemize}
  \item 对力矩的约束对应于必要条件,因此,
  通常,人们只能获得 OCP 最优值的下限;
  \item with $\hat r > r$ and $\hat k > k$,$r$ and $k$ 的原始问题的约束是 $\hat r$ and $\hat{k}$。 因此,增加 $r$ 和 $k$ 的值会产生最优值下限的单调非递减序列。
  \end{itemize}

\begin{remark}
  如果初始测量 $\mu_I$ 未知,我们将不得不包括额外的约束 $M(\w_r)\succeq 0$, $L_{g_{I_j}}(\w_r)\succeq 0$, $\forall j=1,\dots,n_I$ 现在 $\w_r$ 是一个未知的力矩向量,第一个条目等于 1。
\end{remark}

\begin{remark}
  本文的目标之一是从真正基本的概念出发推导 OCP 的凸松弛。 使用紧集 $\K$ 上有界连续函数的 Banach 空间与 $\K$ 上的有限符号 Borel 测度的 Banach 空间之间的对偶性作为起点,也可以获得相同的优化问题,如 \cite{LasPriHen2005,LasHenPriTre2008},其中最优值的下限序列显示在问题数据的某些假设下收敛。 感兴趣的读者可以参考这些论文以了解更多详细信息。
\end{remark}

\section{总结}

本文是 \cite{LasPriHen2005,LasHenPriTre2008} 的后续论文,其中根据占用度量方法推导了多项式最优控制问题 (OCP) 的最优值的下限序列。 在当前的论文中,我们提出了一些技术,可以从 OCP 的凸线性矩阵不等式 (LMI) 松弛的解中建设性地推导出控制律。 因此,我们的贡献可以看作是对 \cite{LasPriHen2005,LasHenPriTre2008} 性能分析结果综合的扩展。

一般来说,我们认为 OCP 的矩公式是基于 Lyapunov 或 Hamilton-Jacobi-Bellman 技术的间接方法的一种有吸引力的替代方法。 力矩公式直接处理系统轨迹。 由此产生的原始 LMI 矩问题承认对偶 LMI 平方和 (SOS) 公式,但是,它有助于显式计算控制律。 在这种情况下,泛函分析(测度论)和代数几何(半代数正多项式的表示)之间的良好相互作用可以为潜在的困难控制综合问题提供建设性的答案。

该方法的当前局限性如下。

首先,当我们正在寻找一个多项式值函数(Hamilton-Jacobi-Bellman 方程的平滑子解)时,它近似于(可能是非平滑的)最优值函数 $\bar{\varphi}(t,x)$,它 可能会发生精度在 $\bar{\varphi}(t,x)$ 不平滑的点处恶化的情况。 沿着最优轨迹移动的邻域中的状态空间划分和/或多项式值函数的迭代计算可以帮助解决这个问题,但代价是增加了计算负担。

其次,我们依赖于当前可用的通用 SDP 求解器的性能。 半定规划是一个相对年轻的研究领域,SDP 求解器的成熟度远不及线性或凸二次规划求解器。 更具体地说,据我们所知,目前没有数值稳定的 SDP 求解器,也没有易于处理的 LMI 问题条件估计。 例如,预计选择表示多项式和矩的基会对问题条件产生重大影响,从而影响求解器的数值行为。

第三,LMI 问题中的变量和约束的数量随着状态和输入变量的数量以及价值函数的多项式逼近度的函数而快速增长。 当前的通用 SDP 求解器可以处理数千个变量和约束,远低于对应于具有 6 个状态和 2 个输入的 OCP 的矩 LMI 问题的维度。 出于这些原因,专门针对矩 LMI 问题的拟 Hankel 或拟 Toeplitz 结构量身定制的专用原始对偶内点方法将受到欢迎。

最后,我们目前正在为 GloptiPoly 3 \cite{HenLasLof2007} 开发一个用户友好的 OCP 模块,这有助于明确地将 OCP 表述为广义矩问题。 用户只需提供OCP的多项式数据,模块自动生成近似最优控制律。 一旦准备就绪并完整记录,该软件将可从 GloptiPoly 3 网页免费下载。

% \tableofcontents


% 本科生的外文资料书面翻译。


% \section{图表示例}

% \subsection{图}

% 附录中的图片示例(图~\ref{fig:appendix-translation-figure})。

% \begin{figure}
%   \centering
%   \includegraphics[width=0.6\linewidth]{example-image-a.pdf}
%   \caption{附录中的图片示例}
%   \label{fig:appendix-translation-figure}
% \end{figure}


% \subsection{表格}

% 附录中的表格示例(表~\ref{tab:appendix-translation-table})。

% \begin{table}
%   \centering
%   \caption{附录中的表格示例}
%   \begin{tabular}{ll}
%     \toprule
%     文件名          & 描述                         \\
%     \midrule
%     thuthesis.dtx   & 模板的源文件,包括文档和注释 \\
%     thuthesis.cls   & 模板文件                     \\
%     thuthesis-*.bst & BibTeX 参考文献表样式文件    \\
%     thuthesis-*.bbx & BibLaTeX 参考文献表样式文件  \\
%     thuthesis-*.cbx & BibLaTeX 引用样式文件        \\
%     \bottomrule
%   \end{tabular}
%   \label{tab:appendix-translation-table}
% \end{table}


% \section{数学公式}

% 附录中的数学公式示例(公式\eqref{eq:appendix-translation-equation})。
% \begin{equation}
%   \frac{1}{2 \uppi \symup{i}} \int_\gamma f = \sum_{k=1}^m n(\gamma; a_k) \mathscr{R}(f; a_k)
%   \label{eq:appendix-translation-equation}
% \end{equation}


% \section{文献引用}

% 文献引用示例\cite{abrahams99tex}。


% \appendix

% \section{附录}

% 附录的内容。


% % 书面翻译的参考文献
\bibliographystyle{unsrtnat}
\bibliography{ref/appendix}

% % 书面翻译对应的原文索引
% \begin{translation-index}
%   \nocite{salomon1995advanced}
%   \bibliographystyle{unsrtnat}
%   \bibliography{ref/appendix}
% \end{translation-index}


\end{translation}
  % 本科生:外文资料的书面翻译
% !TeX root = ../thuthesis-example.tex

\chapter{补充内容}

\section{性质~\ref{prop:boundeddual}的证明}
\label{app:proof:prop:boundeddual}

\begin{proof}
  (1)考虑$\mathbb{U}$是一个多面体,且其没有退化的极值点。观察可得,“$\max_{u \in \mathbb{U}} L_gb(x, \theta) u$”是一个关于$u$的线性函数(当$(x, \theta)$固定的时候)。由于$\mathbb{U}$没有退化的极值点,在最优解$u^\star$处,必有$m' \le m$个线性无关的约束被激活。因此,共有$m'$个非零的$\zeta_i^\star$~\cite{bertsimas97book-lp}。不失一般性地,我们假设$i = 1, \dots, m'$为被激活的约束。则KKT的stationary条件~\eqref{eq:kktstationary}可以被表示为:
  \begin{eqnarray}
    \underbrace{
      \left[ \begin{matrix}
        w_1 & \cdots & w_{m'}
      \end{matrix} \right]
    }_{:= W \in \mathbb{R}^{m \times m'}} = L_gb(x, \theta)
  \end{eqnarray}
  而这暗含了$(\zeta^\star)^T (W^TW) (\zeta^\star) = \parallel L_gb(x, \theta) \parallel^2$。当$W^TW \succ 0$时(亦即,$\left\{ w_i \right\}_{i=1}^{m'}$线性无关),我们有
  \begin{eqnarray}
    \left\lVert \zeta^\star \right\rVert \le \frac{
      \left\lVert L_gb(x, \theta) \right\rVert^2
    }{
      \lambda_{\min} (W^T W)
    }
  \end{eqnarray}
  最后,我们观察到$\left\lVert L_gb(x, \theta) \right\rVert^2$是有界的。这是因为它在紧集$\partial \mathcal{C} \times \Theta$上是光滑的。因此,$\left\lVert \zeta^\star \right\rVert^2$是有界的。

  (2)考虑$\mathbb{U}$是一个箱型,它的每一个维度的取值范围都是$[-w_i, w_i]$。如此一来,KKT条件中的complementarity条件~\eqref{eq:kktcomp}说明了以下两者中有且仅有一者成立:第一,$\zeta_i^\star = 0$;第二,$\zeta_i^\star \ne 0$,但是$u_i^\star = \pm w_i$。在第二种情形下,使用KKT条件中的stationary条件~\eqref{eq:kktstationary},我们有
  \begin{eqnarray}
    \zeta_i^\star = \frac{
      [L_gb(x, \theta)]_i
    }{2u_i^\star} \Rightarrow 
    (\zeta_i^\star)^2 = \frac{
      [L_gb(x, \theta)]_i^2
    }{4 w_i^2}, i = 1, \dots, m
  \end{eqnarray}
  其中$[L_gb(x, \theta)]_i$表示了$L_gb(x, \theta)$的第$i$项。由于$[L_gb(x, \theta)]_i$是有界的,$(\zeta_i^\star)^2$也是有界的。

  (3)现考虑$\mathbb{U}$是一个椭球,且其被一个单独的约束$u^T W u \le 1$所定义。则KKT条件的complentarity条件~\eqref{eq:kktcomp}说明了以下两者中有且仅有一者成立:第一,$\zeta^\star=0$;第二,$\zeta^\star \ne 0$,但是$(u^\star)^T W (u^\star) = 1$。在第二种条件的情况下,使用KKT条件中的stationary条件~\eqref{eq:kktstationary},我们有
  \begin{eqnarray}
    (\zeta^\star)^2 = \frac{
      \left\lVert L_gb(x, \theta) \right\rVert^2
    }{4 \left\lVert W u^\star \right\rVert^2} 
    \le \frac{
      \left\lVert L_gb(x, \theta) \right\rVert^2
    }{4 \lambda_{\min}(W)} 
  \end{eqnarray}
  由于$\left\lVert L_gb(x, \theta) \right\rVert^2$在$\partial \mathcal{C} \times \Theta$上是有界的,$(\zeta^\star)^2$在$\partial \mathcal{C} \times \Theta$上也是有界的。
\end{proof}

\section{对于系统~\eqref{eq:cleanvdpodynamics}和控制障碍函数~\eqref{eq:cleanvdpocbf},证明~\eqref{eq:standardpop}中的$y$有界}
\label{app:bound:y:cleanvdp}

明显地,$\left\lVert x \right\rVert^2 \le \theta$,$u^2 \le u_{\max}^2$是有界的。然而,我们还需要证明$\zeta^\star$是有界的。KKT条件的complentarity条件~\eqref{eq:kktcomp}说明了以下两者中有且仅有一者成立:第一,$\zeta^\star=0$;第二,$\zeta^\star \ne 0$,但是$c_{u,i}(u) = 0$,这意味着$u^\star = \pm u_{\max}$。在第二种情况下,使用KKT条件中的stationary条件~\eqref{eq:kktstationary},我们有
\begin{eqnarray}
  2\zeta^\star u^\star = L_gb(x, \theta) = -2 x_1 x_2
\end{eqnarray}
求解其中的$\zeta^\star$,我们有
\begin{eqnarray}
  \zeta = \frac{-x_1 x_2}{u^\star} \Rightarrow 
  (\zeta^\star)^2 = \frac{x_1^2 x_2^2}{u_{\max}^2} \le \frac{
    (x_1^2 + x_2^2)^2
  }{4 u_{\max}^2} = \frac{\theta_2}{4 u_{\max}^2}
\end{eqnarray}

\section{性质~\ref{prop:wrongvdpocbf}的证明}
\label{app:proofwrongvdpocbf}

\begin{proof}
  我们有
  \begin{subequations}
    \begin{eqnarray}
      L_fb(x, \theta) =& -x_2^2 (1 - x_2^2) \\
      L_gb(x, \theta) =& -2x_1 x_2 \\
      L_Jb(x, \theta) =& -2x_2
    \end{eqnarray}
  \end{subequations}
  作为结果,我们有:
  \begin{subequations}
    \begin{eqnarray}
      V_u^\star =& \displaystyle \max_{u^2 \le u_{\max}^2} (-2 x_1 x_2) u 
      = 2 u_{\max} |x_1 x_2| \\
      V_\epsilon^\star =& \displaystyle \min_{
        \left\lVert \epsilon \right\rVert \le M_\epsilon
      } -2 x_2 \epsilon 
      = -2 M_\epsilon |x_2|
    \end{eqnarray}
  \end{subequations}
  以及
  \begin{eqnarray}
    V(\theta) = \min_{\left\lVert x \right\rVert^2 = \theta} 
    -x_2^2 (1 - x_2^2) + u_{\max} |x_1 x_2| - M_\epsilon |x_2|
  \end{eqnarray}
  我们选择$x_2 = 0, x_1 = \pm \sqrt{\theta}$,可以得到$V(\theta) \le 0$。
\end{proof}

\section{对于系统~\eqref{eq:vdpodynamics}和控制障碍函数~\eqref{eq:ellipsoidalcbf},证明~\eqref{eq:standardpop}中的$y$有界}
\label{app:bound:y:uncertainvdp}

显然,$u^2 \le u_{\max}$是有界的。而$x$也是有界的,这是因为
\begin{eqnarray}
  \max_{x^T A x = 1} \overset{v := A^{1/2}x}{=} 
  \max_{v^T v = 1} v^T A^{-1} v = \frac{1}{\lambda_{\min}(A)}
\end{eqnarray}
其中$\lambda_{\min}$和$\lambda_{\max}$表示了$A$的特征值中的最大值和最小值。注意到
\begin{eqnarray}
  \lambda_{\min} (\lambda_{\max} + \lambda_{\min}) \ge \lambda_{\min} \lambda_{\max} = \theta_1 \theta_2 - \theta_3^2 \Rightarrow \\
  \lambda_{\min} \ge \frac{
    \theta_1 \theta_2 - \theta_3^2
  }{
    \lambda_{\min} + \lambda_{\max}
  } = \frac{
    \theta_1 \theta_2 - \theta_3^2
  }{
    \theta_1 + \theta_2
  }
\end{eqnarray}
因此,$x$是有界的,因为
\begin{eqnarray}
  \left\lVert x \right\rVert^2 \le \frac{1}{\lambda_{\min}(A)} 
  \le \frac{
    \theta_1 + \theta_2
  }{
    \theta_1 \theta_2 - \theta_3^2
  }
\end{eqnarray}
现在考虑$z^2 = \left\lVert L_Jb(x, \theta) \right\rVert^2$:
\begin{eqnarray}
  \left\lVert L_Jb(x, \theta) \right\rVert^2 = 
  \left\lVert -2 x^T A J(x) \right\rVert^2 = 
  4 x^T A J(x)J(x)^T A x
\end{eqnarray}
注意到$J(x)J(x)^T \preceq \textbf{I}$,这就意味着
\begin{eqnarray}
  \left\lVert L_Jb(x, \theta) \right\rVert^2 \le 
  4x^T A^2 x \le 4 \lambda_{\max}(A) \le 4(\theta_1 + \theta_2)
\end{eqnarray}
现在还需要证明$\zeta^\star$也是有界的。与附录~\ref{app:bound:y:cleanvdp}相似,KKT条件能够告诉我们以下两者中有且仅有一种成立:第一,$\zeta^\star = 0$;第二,
\begin{eqnarray}
  \zeta^\star = \frac{L_gb(x, \theta)}{2u^\star}, \quad (u^\star)^2 = u_{\max}^2
\end{eqnarray}
我们将$(L_gb(x, \theta))^2$写为
\begin{eqnarray}
  (L_gb(x, \theta))^2 = 
  \left\lVert -2 x^T Ag \right\rVert^2 = 4 x^T Agg^T Ax
\end{eqnarray}
其中$gg^T \preceq (x_1^2 + x_2^2) \textbf{I}$,我们可以得到
\begin{eqnarray}
  (L_gb(x, \theta))^2 \le 
  4 \left\lVert x \right\rVert^2 x^T A^2 x \le 
  4 \frac{
    (\theta_1 + \theta_2)^2
  }{
    \theta_1 \theta_2 - \theta_3^2
  } \\
  \Rightarrow (\zeta^\star)^2 = 
  \frac{
    (L_gb(x, \theta))^2
  }{4 u_{\max}^2} \le 
  \frac{
    (\theta_1 + \theta_2)^2
  }{
    u_{\max}^2 (\theta_1 \theta_2 - \theta_3^2)
  }
\end{eqnarray}

\section{计算矩}\label{sec:app:computemoments}

\subsection{无不确定性的受控Van der Pol振荡器}
在这种情况下,我们使用的是圆形控制障碍函数。考虑参数空间$\Theta \in [\theta_{\min}, \theta_{\max}]$,令$\Delta_\theta = \theta_{\max} - \theta_{\min}$成为$\Theta$的“长度”。我们计算\eqref{eq:computemoments}的矩
\begin{eqnarray}
  \gamma_\beta = \int_{\theta_{\min}}^{\theta_{\max}} \theta^\beta d \psi (\theta) = 
  \frac{1}{\Delta_\theta} \int_{\theta_{\min}}^{\theta_{\max}} \theta^\beta d\theta = 
  \frac{
    \theta_{\max}^{\beta+1} - \theta_{\min}^{\beta+1}
  }{(\beta + 1) \Delta_\theta}
\end{eqnarray}

\subsection{具备不确定性的受控Van der Pol振荡器}
这种情况下,我们使用椭圆形控制障碍函数。对于~\eqref{eq:ellipsoidalparam}所定义的$\Theta$,我们计算~\eqref{eq:computemoments}:
\begin{eqnarray}
  & \gamma_\beta = \displaystyle \int_{\Theta} \theta_1^{\beta_1} \theta_2^{\beta_2} \theta_3^{\beta_3} d \psi(\theta) \nonumber \\
  = & \displaystyle \int_{\theta_1} \int_{\theta_2} \int_{\theta_3} \theta_1^{\beta_1} \theta_2^{\beta_2} \theta_3^{\beta_3} \left(  
    \frac{1}{\text{vol}\Theta} d \theta_1 d \theta_2 d \theta_3
    \right) \nonumber \\
  = & \displaystyle \frac{1}{\text{vol}\Theta} \int_{\theta_1} \int_{\theta_2} \theta_1^{\beta_1} \theta_2^{\beta_2} \left(
    \int_{-\xi \sqrt(\theta_1 \theta_2)}^{\xi \sqrt(\theta_1 \theta_2)} \theta_3^{\beta_3} d \theta_3
    \right) \nonumber \\
  = & \displaystyle \frac{
    \xi^{\beta_3 + 1} (1 - (-1)^{\beta_3 + 1})
  }{
    (\beta_3 + 1) \text{vol}(\Theta)
  } \int_{\theta_1} \int_{\theta_2} 
  \theta_1^{\beta_1 + \frac{\beta_3}{2}} 
  \theta_2^{\beta_2 + \frac{\beta_3}{2}} d \theta_1 d \theta_2 \nonumber \\
  = & \displaystyle \frac{
    \xi^{\beta_3 + 1} (1 - (-1)^{\beta_3 + 1})
  }{
    (\beta_3 + 1) \text{vol}(\Theta)
  } \frac{
    \bar{\theta}^{\tilde{\beta}_1} - \underbar{\theta}^{\tilde{\beta}_1}
  }{\tilde{\beta}_1}
  \frac{
    \bar{\theta}^{\tilde{\beta}_2} - \underbar{\theta}^{\tilde{\beta}_2}
  }{\tilde{\beta}_2}
\end{eqnarray}
其中,$\text{\Theta}$是$\Theta$的体积(一个常数,我们不需要关心)。而$\tilde{\beta}_1 = \beta_1 + \frac{\beta_3}{2}$,$\tilde{\beta}_2 = \beta_2 + \frac{\beta_3}{2}$。

% 致谢
\input{data/acknowledgements}

% 声明
\statement
% 将签字扫描后的声明文件 scan-statement.pdf 替换原始页面
% \statement[file=scan-statement.pdf]
% 本科生编译生成的声明页默认不加页脚,插入扫描版时再补上;
% 研究生编译生成时有页眉页脚,插入扫描版时不再重复。
% 也可以手动控制是否加页眉页脚
% \statement[page-style=empty]
% \statement[file=scan-statement.pdf, page-style=plain]

% 个人简历、在学期间完成的相关学术成果
% 本科生可以附个人简历,也可以不附个人简历
% \input{data/resume}

% 指导教师/指导小组评语
% 本科生不需要
% \input{data/comments}

% 答辩委员会决议书
% 本科生不需要
% \input{data/resolution}

% 本科生的综合论文训练记录表(扫描版)
% \record{file=scan-record.pdf}

\end{document}
