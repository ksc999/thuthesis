% !TeX root = ../thuthesis-example.tex

\chapter{多控制障碍函数情形}

本章节旨在将没有不确定性的多个控制障碍函数的验证问题转化为单级多项式优化问题。

\section{多控制障碍函数问题引入}
\begin{definition}[多控制障碍函数安全集合定义] \label{def:safe-set-multi-cbf}
给定$M$个函数$\left\{ b_i(x),i=1\cdots M \right\} $,安全集合$\mathcal{C}$被定义为$\mathcal{C} =\left\{ x\mid b_i\left( x \right) \ge 0,\forall i=1\cdots M \right\} $. 
\end{definition}

根据定义~\ref{def:safe-set-multi-cbf},我们能够刻画$\mathcal{C}$的边界:
\begin{equation}    \label{eq:safe-set-boundary-multi-cbf}
\begin{aligned}
	\partial \mathcal{C} &=\left\{ x\left| \begin{array}{c}
	x\in \mathbb{X}\\
	\min \left\{ b_1\left( x \right) \cdots b_M\left( x \right) \right\} =0\\
\end{array} \right. \right\}\\
	&=\left\{ x\left| \begin{array}{c}
	x\in \mathbb{X}\\
	\exists t\left( \cdot \right) ,\exists k\in \left[ 1,M \right] , s.t.\begin{cases}
	b_{t\left( i \right)}\left( x \right) =0,\forall i=1\cdots k\\
	b_{t\left( i \right)}\left( x \right) >0,\forall i=k+1\cdots M\\
\end{cases}\\
\end{array} \right. \right\}\\
\end{aligned}
\end{equation}
其中$t(\cdot)$是一个排列函数,例如$t\left( \left\{ 1,2,3 \right\} \right) =\left\{ 3,1,2 \right\} $注意到对于一个给定的$x \in \partial \mathcal{C}$,
\begin{equation}    \label{eq:k-multi-cbf}
k=k\left( x \right) =\#\left\{ i\mid b_i(x)=0 \right\} 
\end{equation}
是唯一的。然而,$t(\cdot)$的选择就更加灵活,我们只需:
\begin{equation}    \label{eq:t-multi-cbf}
t(\cdot )\in T\left( x \right) =\left\{ t(\cdot )\mid \begin{array}{c}
	t(j)\in \left\{ i\mid b_i(x)=0 \right\} ,\forall j\in \{1\cdots k\}\\
	t(j)\in \left\{ i\mid b_i(x)>0 \right\} ,\forall j\in \{k+1\cdots M\}\\
\end{array} \right\} 
\end{equation}

\begin{definition}[合法的多控制障碍函数族] \label{def:valid-multi-cbf}
 $\left\{ b_i(x),i=1\cdots M \right\} $是一个合法的多控制障碍函数族,当且仅当$\forall x\in \partial \mathcal{C} $, $\exists u_0\in \mathbb{U}$,使得$\dot{b}_{t\left( i \right)}\left( x,u_0 \right) \ge 0,\forall i=1\cdots k$。其中$k=k(x)$和$t(\cdot) \in T(x)$从~\eqref{eq:k-multi-cbf}和~\eqref{eq:t-multi-cbf}中得来。同样地,$V_0$在~\eqref{eq:V0-multi-cbf}中,我们希望$V_0 \ge 0$。
\end{definition}

\begin{equation} \label{eq:V0-multi-cbf}
V_0=\min _{x\in \partial \mathcal{C}}\max _{u\in \mathbb{U}}\min _{i\in \left\{ 1\cdots k \right\}}\dot{b}_{t\left( i \right)}\left( x,u \right) 
\end{equation}

这是一个三级的多项式优化问题,很难解决。 在\ref{sec:semi-C-multi-cbf} 节中,我们将$\partial \mathcal{C}$表示为一个半代数集。 在第\ref{sec:single-POP-multi-cbf}节中,我们展示了$u$上的最大值和$i$上的最小值可以以某种方式交换,这将导致一个两级的可解多项式优化问题。
\begin{equation} \label{eq:V0-prime-multi-cbf}
V_{0}^{'}=\min_{x\in \partial \mathcal{C}} \min_{i\in \left\{ 1\cdots k \right\}} \max_{u\in \mathbb{U}} \dot{b}_{t\left( i \right)}\left( x,u \right) 
\end{equation}

\section{$\partial \mathcal{C}$的基本半代数表示}
\label{sec:semi-C-multi-cbf}
我们首先使用一个半代数集来表示 $\partial \mathcal{C}$。 将 $\mathcal{A}$ 定义为一组 $\left( x,\overrightarrow{\alpha }=\left\{ \alpha _1\cdots \alpha _M \right\} \right)$,其中 $x$ 和 $\overrightarrow{\alpha}$ 满足以下约束:
\begin{eqnarray}
x \in \mathbb{X}\\
\sum_{i=1}^M{\alpha _ib_i\left( x \right)} = 0 \label{eq:x-alpha-zero-eq-multi-cbf}\\    
b_i\left( x \right) \ge 0,\forall i=1\cdots M \label{eq:x-alpha-b-ineq-multi-cbf}\\
\alpha _i \ge 0,\forall i=1\cdots M \label{eq:x-alpha-alpha-ineq-multi-cbf}\\   
\sum_{i=1}^M{\alpha _i} = 1 \label{eq:x-alpha-alpha-sum-multi-cbf}
\end{eqnarray}

紧接着,$\partial \mathcal{C}$和$\mathcal{A}$有着以下紧密联系:
\begin{lemma} \label{lemma:semi-alge-bound-multi-cbf}
    在~\eqref{eq:hierachical-bound-multi-cbf}中定义集合$\mathcal{B}$,则$\mathcal{A} = \mathcal{B}$。 这里 $k=k(x)$ 和 $t(\cdot) \in T(x)$ in~\eqref{eq:hierachical-bound-multi-cbf} 来自~\eqref{eq:k-multi-cbf} 和 ~\eqref{eq:t-multi-cbf}。
\end{lemma}

\begin{equation} \label{eq:hierachical-bound-multi-cbf}
\mathcal{B} =\left\{ \left( x,\overrightarrow{\alpha } \right) \left| x\in \partial \mathcal{C} , \text{ for the given }x:\begin{cases}
	\alpha _{t\left( i \right)}\ge 0,i=1\cdots k\\
	\sum_{i=1}^k{\alpha _{t\left( i \right)}}=1\\
	\alpha _{t\left( i \right)}=0,i=k+1\cdots M\\
\end{cases} \right. \right\} 
\end{equation}

\begin{remark}
    引理~\ref{lemma:semi-alge-bound-multi-cbf} 暗示 $\mathcal{A}$ 实际上有一个层次结构: (1) 它存储所有 $x \in \partial \mathcal{C}$ . (2) 给定 $x \in \partial \mathcal{C}$,$k=k(x)$ 和 $t(\cdot) \in T(x)$ 是明确定义的。 然后 $\overrightarrow{\alpha}$ 可以从 ~\eqref{eq:hierachical-alpha-multi-cbf} 中定义的集合中选择。
\end{remark}

\begin{proof}
    
(1) $\mathcal{A} \subseteq \mathcal{B}$:

对于 $\forall \left( x,\overrightarrow{\alpha } \right) \in \mathcal{A}$,从 ~\eqref{eq:x-alpha-zero-eq-multi-cbf} 到 ~\eqref{eq:x-alpha-alpha-sum-multi-cbf},我们知道至少存在一个$i \in \left\{ 1 \cdots M \right\}$,使得$b_i(x) = 0$。否则,如果 $b_i(x) > 0, \forall i=1\cdots M$,则从 ~\eqref{eq:x-alpha-zero-eq-multi-cbf}, $\alpha_i = 0, \forall i=1\cdots M$。 然而,在这种情况下,$\Sigma \alpha_i = 0$,这违反了~\eqref{eq:x-alpha-alpha-sum-multi-cbf}。

上面已经暗示了 $\min \left\{ b_1(x) \cdots b_M(x) \right\} = 0$。 因此,从 ~\eqref{eq:safe-set-boundary-multi-cbf} 的第一个方程,$x \in \partial \mathcal{C}$。 然后我们可以在这个 $x$ 上定义 $k=k(x)$ 和 $t(\cdot) \in T(x)$。 由于
\begin{equation}
\begin{cases}
	\alpha _{t\left( i \right)}\ge 0,b_{t\left( i \right)}\left( x \right) =0,i=1\cdots k\\
	\alpha _{t\left( i \right)}=0,b_{t\left( i \right)}\left( x \right) >0,i=k+1\cdots M\\
\end{cases}
\end{equation}
下述的约束天然成立:
\begin{equation} \label{eq:hierachical-alpha-multi-cbf}
\begin{cases}
	\alpha _{t\left( i \right)}\ge 0,i=1\cdots k\\
	\sum_{i=1}^k{\alpha _{t\left( i \right)}}=1\\
	\alpha _{t\left( i \right)}=0,i=k+1\cdots M\\
\end{cases}
\end{equation}

(2) $\mathcal{A} \supseteq \mathcal{B}$:

对于 $\forall (x, \overrightarrow{\alpha}) \in \mathcal{B}$,~\eqref{eq:x-alpha-alpha-sum-multi-cbf} 和~\eqref{eq:x-alpha-alpha-ineq-multi-cbf} 自动满足。 此外,在这种情况下,
\begin{equation}
\begin{aligned}
	\sum_{i=1}^M{\alpha _ib_i\left( x \right)}&=\sum_{i=1}^k{\alpha _{t\left( i \right)}b_{t\left( i \right)}\left( x \right)}+\sum_{i=k+1}^M{\alpha _{t\left( i \right)}b_{t\left( i \right)}\left( x \right)}\\
	&=\sum_{i=1}^k{\alpha _{t\left( i \right)}\cdot 0}+\sum_{i=k+1}^M{0\cdot b_{t\left( i \right)}\left( x \right)}=0\\
\end{aligned}
\end{equation}
因此,~\eqref{eq:x-alpha-zero-eq-multi-cbf} 是满足的。 此外,由于 $\min \left\{ b_1(x) \cdots b_M(x) \right\} = 0$,~\eqref{eq:x-alpha-b-ineq-multi-cbf} 自动满足。 根据 $\mathcal{A}$ 的定义,存在 $\overrightarrow{\alpha}$, s.t. $\left( x,\overrightarrow{\alpha } \right) \in \mathcal{A}$。

\end{proof}

\section{多控制障碍函数验证问题的单级多项式优化问题表述}
\label{sec:single-POP-multi-cbf}
现在,我们简历双极多项式优化问题来表征验证问题:
\begin{equation}    \label{eq:V-multi-cbf}
V=\min_{\left( x,\overrightarrow{\alpha } \right) \in \mathcal{A}}\max_{u\in \mathbb{U}} \sum_{i=1}^M{\alpha _i\dot{b}_i\left( x,u \right)}
\end{equation}
在我们陈述 $V_0$ 和 $V$ 之间的关系之前,让我们先看看下面的引理。

\begin{lemma} \label{lemma:H1-H2-multi-cbf}
    给定来自 ~\eqref{eq:H1-H2-def-multi-cbf} 的 $H_1$ 和 $H_2$,其中 $u\in \mathbb{R} ^d,A_i\in \mathbb{R} ,B_i\ 在 \mathbb{R} ^{1\times d}$ 和凸集 $\mathbb{U} =\left\{ c_w\left( u \right) \le 0,w=1\cdots h \right\} $。 我们有 $H_1 = H_2$。
\end{lemma}

\begin{eqnarray} \label{eq:H1-H2-def-multi-cbf}
	H_1=\max_{u\in \mathbb{U}}\min_{i\in \left\{ 1\cdots l \right\}}\left( A_i+B_iu \right)\\
	H_2=\min_{\begin{array}{c}
	\overrightarrow{\alpha }\succeq 0\\
	\Sigma \alpha _i=1\\
\end{array}}\max_{u\in \mathbb{U}}\sum_{i=1}^l{\alpha _i\left( A_i+B_iu \right)}
\end{eqnarray}

\begin{remark}
    引理~\ref{lemma:H1-H2-multi-cbf} 成立是因为 $H_1$ 中的内部最小值定义了线性函数上的逐点最小值,这导致凹函数。 那么,$H_1$是一个凸优化问题,我们可以使用KKT条件来处理它。 此外,$\overrightarrow{\alpha}^{\star}$ from $arg \min H_2$ 具有明确的含义:它是当 $H_1$ 达到最佳值时逐点最小子梯度的系数,如 以下证明。
\end{remark}

\begin{proof}
    
(1) $H_1 \le H_2$.。

使用最小-最大不等式:
\begin{equation}
\min_{\begin{array}{c}
	\overrightarrow{\alpha }\succeq 0\\
	\Sigma \alpha _i=1\\
\end{array}}\max_{u\in \mathbb{U}}\sum_{i=1}^l{\alpha _i\left( A_i+B_iu \right)}\ge \max_{u\in \mathbb{U}}\min_{\begin{array}{c}
	\overrightarrow{\alpha }\succeq 0\\
	\Sigma \alpha _i=1\\
\end{array}}\sum_{i=1}^l{\alpha _i\left( A_i+B_iu \right)}
\end{equation}
但对于一个给定的$u \in \mathbb{U}$,我们同样有:
\begin{equation}
\min_{\begin{array}{c}
	\overrightarrow{\alpha }\succeq 0\\
	\Sigma \alpha _i=1\\
\end{array}}\sum_{i=1}^l{\alpha _i\left( A_i+B_iu \right)}=\min_{i\in \left\{ 1\cdots l \right\}}\left( A_i+B_iu \right) 
\end{equation}
因此,$H_1 \le H_2$。

(2)$H_1 \ge H_2$。

定义$f(u)$如下:
\begin{equation}
f\left( u \right) =\max_{i\in \left\{ 1\cdots l \right\}}-\left( A_i+B_iu \right) 
\end{equation}
它是一组线性函数的逐点最大值。 因此,$f(u)$ 是一个凸函数。 然后我们将 $H_1$ 重写为单级凸优化问题:
\begin{equation}
\begin{aligned}
	H_1&=\max_{u\in \mathbb{U}}\min_{i\in \left\{ 1\cdots l \right\}}\left( A_i+B_iu \right)\\
	&=-\min_{u\in \mathbb{U}}\max_{i\in \left\{ 1\cdots l \right\}}-\left( A_i+B_iu \right) =-\min_{u\in \mathbb{U}}f\left( u \right)\\
\end{aligned}
\end{equation}

虽然 $f(u)$ 不是连续函数,我们仍然可以从~\href{https://people.orie.cornell.edu/mru8/orie6326/lectures/subgradient.pdf}{子梯度讲义中的46页}:
\begin{equation} \label{eq:fu-subgrad-multi-cbf}
\partial f\left( u \right) =\left\{ -\sum_{i=1}^l{\alpha _iB_i}\left| \begin{array}{c}
	\overrightarrow{\alpha }\succeq 0\\
	\sum_{i=1}^l{\alpha _i}=1\\
    \alpha _i=0 \quad if\,\,-\left( A_i+B_iu \right) <f\left( u \right) \\
\end{array} \right. \right\}
\end{equation}
紧接着,我们写出$H_1$的KKT条件:
\begin{eqnarray}
	0\in \partial f\left( u^{\star} \right) +\sum_{j=1}^h{\zeta _{j}^{\star}\frac{\partial c_j\left( u^{\star} \right)}{\partial u}} \label{eq:H1-KKT-stationary-multi-cbf} \\
	c_j\left( u^{\star} \right) \le 0, j=1\cdots h\\
	\zeta _{j}^{\star}\ge 0, j=1\cdots h\\
	\zeta _{j}^{\star}c_j\left( u^{\star} \right) =0, j=1\cdots h \label{eq:H1-KKT-complementary-multi-cbf}
\end{eqnarray}
将~\eqref{eq:H1-KKT-stationary-multi-cbf}中的$\partial f(u^{\star})$替换为~\eqref{eq:fu-subgrad-multi-cbf},我们可以得到 一个新的静止条件:
\begin{equation}
\exists \overrightarrow{\alpha }^{\star}, s.t. \begin{cases}
	-\sum_{i=1}^l{\alpha _{i}^{\star}B_i}++\sum_{j=1}^h{\zeta _{j}^{\star}\frac{\partial c_j\left( u^{\star} \right)}{\partial u}}=0\\
	\overrightarrow{\alpha }^{\star}\succeq 0\\
	\sum_{i=1}^l{\alpha _{i}^{\star}}=1\\
	\alpha _{i}^{\star}=0 \quad if\,\,-\left( A_i+B_iu^{\star} \right) <f\left( u^{\star} \right)\\
\end{cases}
\end{equation}
我们归一化$H_2$,使用的工具为$\overrightarrow{\alpha}^{\star}$:
\begin{equation}
\begin{aligned}
	H_2&=\min_{\begin{array}{c}
	\overrightarrow{\alpha }\succeq 0\\
	\Sigma \alpha _i=1\\
\end{array}}\max_{u\in \mathbb{U}}\sum_{i=1}^l{\alpha _i\left( A_i+B_iu \right)}\\
	&\le \max_{u\in \mathbb{U}}\sum_{i=1}^l{\alpha _{i}^{\star}\left( A_i+B_iu \right)}\\
	&=-\min_{u\in \mathbb{U}}-\left( \sum_{i=1}^l{\alpha _{i}^{\star}A_i} \right) -\left( \sum_{i=1}^l{\alpha _{i}^{\star}B_i} \right) u\\
\end{aligned}
\end{equation}
幸运的是,当我们写 $H_1$ 的 KKT 条件时,它与 $H_2$ 的 KKT 条件完全相同,即 $\left( \overrightarrow{\alpha }^{\star},u^{\star },\zeta ^{\star} \right) $定义在~\eqref{eq:H1-KKT-stationary-multi-cbf} -~\eqref{eq:H1-KKT-complementary-multi-cbf}也可以 导致 $H_2$ 中的原始最优性和对偶最优性(考虑 Slater 条件成立)。
\begin{equation}
\begin{aligned}
	&-\min_{u\in \mathbb{U}} -\left( \sum_{i=1}^l{\alpha _{i}^{\star}A_i} \right) -\left( \sum_{i=1}^l{\alpha _{i}^{\star}B_i} \right) u\\
    &=-\left( \sum_{i=1}^l{\alpha _{i}^{\star}A_i} \right) -\left( \sum_{i=1}^l{\alpha _{i}^{\star}B_i} \right) u^{\star}\\
	&=-\sum_{i=1}^l{\alpha _{i}^{\star}\left( A_i+B_iu^{\star} \right)}\\
\end{aligned}
\end{equation}
由于
\begin{equation}
\begin{cases}
	\alpha _{i}^{\star}=0,-\left( A_i+B_iu^{\star} \right) <f\left( u^{\star} \right)\\
	\alpha _{i}^{\star}\ge 0,-\left( A_i+B_iu^{\star} \right) =f\left( u^{\star} \right)\\
	\sum_{i=1}^l{\alpha _{i}^{\star}}=1\\
\end{cases}
\end{equation}
我们有
\begin{equation}
H_2\le -\sum_{i=1}^l{\alpha _{i}^{\star}\left( A_i+B_iu^{\star} \right)}=f\left( u^{\star} \right) =H_1
\end{equation}
\end{proof}

下面陈述本章的主要理论:
\begin{theorem} \label{thm:V0-V-multi-cbf}
    给定 ~\eqref{eq:V0-multi-cbf} 中的 $V_0$ 和 ~\eqref{eq:V-multi-cbf} 中的 $V$,我们有 $V=V_0$。
\end{theorem}

\begin{proof}
首先,我们注意到给定 $x$,$\dot{b}_i(x, u)$ 在 $u$ 中是线性的:
\begin{equation}
\dot{b}_i\left( x,u \right) =L_fb_i\left( x \right) +L_gb_i\left( x \right) \cdot u
\end{equation}
因此,通过调用引理~\ref{lemma:H1-H2-multi-cbf},给定 $x \in \partial \mathcal{C}$,我们有:
\begin{equation} \label{eq:main-thm-1-multi-cbf}
\max _{u\in \mathbb{U}}\min _{i\in \left\{ 1\cdots k \right\}}\dot{b}_{t\left( i \right)}\left( x,u \right) =\min _{\begin{array}{c}
	\alpha _{t\left( i \right)}\ge 0,i=1\cdots k\\
	\sum_{i=1}^k{\alpha _{t\left( i \right)}}=1\\
\end{array}}\max _{u\in \mathbb{U}}\sum_{i=1}^k{\alpha _{t\left( i \right)}}\dot{b}_{t\left( i \right)}\left( x,u \right)
\end{equation}
其中 $k=k(x)$ 和 $t(\cdot) \in T(x)$ 来自~\eqref{eq:k-multi-cbf} 和~\eqref{eq:t-multi-cbf} . 接下来,为了在 ~\eqref{eq:hierachical-alpha-multi-cbf} 中获得约束,我们手动添加
\begin{equation}
\alpha _{t\left( i \right)}=0,i=k+1\cdots M
\end{equation}
这将导致
\begin{equation}
\begin{aligned}
	\sum_{i=1}^k{\alpha _{t\left( i \right)}}\dot{b}_{t\left( i \right)}\left( x,u \right) &=\sum_{i=1}^k{\alpha _{t\left( i \right)}}\dot{b}_{t\left( i \right)}\left( x,u \right) +\sum_{i=k+1}^M{\alpha _{t\left( i \right)}}\dot{b}_{t\left( i \right)}\left( x,u \right)\\
	&=\sum_{i=1}^M{\alpha _{t\left( i \right)}}\dot{b}_{t\left( i \right)}\left( x,u \right) =\sum_{i=1}^M{\alpha _i}\dot{b}_i\left( x,u \right)\\
\end{aligned}
\end{equation}
和
\begin{eqnarray} \label{eq:main-thm-2-multi-cbf}
& \min_{\begin{array}{c}
	 \alpha_{t\left( i \right)}\ge 0,i=1\cdots k\\
	\sum_{i=1}^k{\alpha_{t\left( i \right)}}=1\\
\end{array}}\max _{u\in \mathbb{U}}\sum_{i=1}^k{\alpha _{t\left( i \right)}}\dot{b}_{t\left( i \right)}\left( x,u \right) \\
= & \min _{\begin{array}{c}
	\alpha _{t\left( i \right)}\ge 0,i=1\cdots k\\
	\sum_{i=1}^k{\alpha _{t\left( i \right)}}=1\\
	\alpha _{t\left( i \right)}=0,i=k+1\cdots M\\
\end{array}}\max _{u\in \mathbb{U}}\sum_{i=1}^M{\alpha _i}\dot{b}_i\left( x,u \right) 
\end{eqnarray}
结合 ~\eqref{eq:main-thm-1-multi-cbf} 和 ~\eqref{eq:main-thm-2-multi-cbf},我们有:
\begin{equation}
\begin{aligned}
	V_0&=\min _{x\in \partial \mathcal{C}}\max _{u\in \mathbb{U}}\min _{i\in \left\{ 1\cdots k \right\}}\dot{b}_{t\left( i \right)}\left( x,u \right)\\
	&=\min _{x\in \partial \mathcal{C}}\min _{\begin{array}{c}
	\alpha _{t\left( i \right)}\ge 0,i=1\cdots k\\
	\sum_{i=1}^k{\alpha _{t\left( i \right)}}=1\\
	\alpha _{t\left( i \right)}=0,i=k+1\cdots M\\
\end{array}}\max _{u\in \mathbb{U}}\sum_{i=1}^M{\alpha _i}\dot{b}_i\left( x,u \right)\\
\end{aligned}
\end{equation}
通过$\mathcal{B}$在~\eqref{eq:hierachical-bound-multi-cbf}中的定义,
\begin{equation}
V_0=\min _{\left( x,\overrightarrow{\alpha } \right) \in \mathcal{B}}\max _{u\in \mathbb{U}}\sum_{i=1}^M{\alpha _i}\dot{b}_i\left( x,u \right) 
\end{equation}
最终,通过引理~\ref{lemma:semi-alge-bound-multi-cbf},
\begin{equation}
V_0=\min _{\left( x,\overrightarrow{\alpha } \right) \in \mathcal{A}}\max _{u\in \mathbb{U}}\sum_{i=1}^M{\alpha _i}\dot{b}_i\left( x,u \right) 
\end{equation}
\end{proof}

现在我们需要证明的是~\eqref{eq:V-multi-cbf} 中定义的二级 POP 问题可以转换为单级 POP。 由于目标函数
\begin{equation}
\sum_{i=1}^M{\alpha _i\dot{b}_i\left( x,u \right)}=\sum_{i=1}^M{\alpha _iL_fb_i\left( x \right)}+\left( \sum_{i=1}^M{\alpha _iL_gb_i\left( x \right)} \right) u
\end{equation}
在 $u$ 中仍然是线性的,我们仍然可以使用 KKT 条件来消除内部最大值。



