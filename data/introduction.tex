% !TeX root = ../thuthesis-example.tex

\chapter{引言}

\section{鲁棒控制障碍函数简介}
安全性在智能与自动系统中至关重要。在控制领域中,基于能量函数的安全控制算法被广泛使用。这一类算法将系统状态约束在某一“安全集合”中。其中,控制障碍函数\cite{ames2014cdc-cbforigin} 引起了越来越多的研究兴趣。控制障碍函数将安全集编码为其控制不变的零上水平集:亦即,如果系统在安全集中启动,则始终存在一控制输入序列,将系统状态保持在这一安全集合中。于此同时,由于动态不确定性在现实世界的应用中无处不在,针对系统不确定性的鲁棒控制障碍函数随之出现,以确保系统在存在不确定性的情况下保持安全。

\section{鲁棒控制障碍函数的验证和综合问题}
在涉及(鲁棒)控制障碍函数的现有文献中,大家将注意力主要集中在CBF的部署问题上:假设已经获得了一个控制障碍函数,在此基础上通过二次规划方法\cite{taylor2020acc-robustqp}或者二阶锥规划\cite{buch21csl-robust}生成一个安全控制器。然而,仍有两个挑战未被解决:如何去验证与综合一个鲁棒安全控制函数?
\begin{enumerate}
    \item 验证问题。若现在给定一个候选鲁棒控制障碍函数,对于其上水平集中包含的所有系统状态以及所有可能的动力学不确定性,我们需要验证是否总是存在一个合法的控制输入,使得系统状态能维持在这一上水平集内。
    \item 综合问题。在一个函数空间中,寻找一个合法的鲁棒控制障碍函数(并且验证它的正确性)。
\end{enumerate}

针对多项式动力学系统以及简单的多面体控制输入边界,现有的工作将综合问题转化为一凸的平方和优化问题~\cite{clark22arxiv-cbf,dai2022arxiv-clfcbfsynveri},并将综合问题转化为一个非凸的双线性平方和优化问题~\cite{wang2018acc-permissive,zhao22arxiv-cbfsos}。在上述方法中,存在两个缺陷。第一,双线性平方和优化算法通过交替下降方法进行优化求解,而该方法无法提供全局收敛性的理论保证。第二,此类方法无法处理动力学模型中的不确定性以及更为广泛的(凸)控制输入约束。

\section{主要贡献}
本文针对的仿射多项式动力学系统具有如下特性:(1)系统模型中的动力学不确定性为有界、加性且与系统状态有关的。(2)系统的控制信号输入被约束在任意的凸包中。在上述模型之上,我们研究鲁棒控制障碍函数的验证和综合问题。首先,我们将上述两个问题进行了符号语言上的统一与公式化:它们均可以被表示为多级多项式优化问题{\color{red} AAA}~\cite{bennett22mp-hierarchical}。特别地,对于一个参数化的鲁棒控制障碍函数候选,验证问题被转化为三级多项式优化问题:我们由外向内,分别优化不确定性,控制信号输入,以及系统状态。综合问题则转换为了一个四级的多项式优化问题,附带额外搜索最佳参数{\color{red} AAA}。尽管多级优化问题难以处理,但本文研究表明,在凸控制约束和有界不确定性这两大前提下,我们可以通过KKT最优性条件以解析解的形式消除内部两级优化(不确定性和控制)。因此,验证问题可以减少到状态上的单级多项式优化问题,综合问题则可以减少到状态和参数上的最小-最大多项式优化问题{\color{red} AAA}。在完成问题表征后,我们采用多级半正定规划松弛来近似求解多项式优化问题,并指出该系列数学处理方法具有全局收敛性保证{\color{red} AAA}。更为具体地,针对由验证问题转化而来的单级多项式优化问题,我们采用Lasserre's Hierarchy这一矩松弛方法~\cite{lasserre01siopt-global}。对于最小-最大多项式优化问题,我们采用双阶段松弛方法{\color{red} AAA}~\cite{lasserre11jgo-minmaxpop}:在第一阶段,我们使用平方和优化方法获得优化问题未知值函数的不断紧逼的多项式下界;在第二阶段,我们再一次通过Lasserre's Hierarchy最大化多项式下界,来搜索鲁棒控制障碍函数的有效参数。读者可以参看图{\color{red} CCC}以直观地感受我们的方法:图中,蓝色虚线是未知值函数,实线是通过平方和优化方法找到的多项式下界。我们对受控Van der Pol振荡器进行了深入研究,并对系统具备和不具备加性不确定性这两种情况分别进行了讨论{\color{red} AAA}。

\section{方法的局限性}
(1)与其它基于半正定松弛的方法类似,我们的方法受到当前解决大规模半正定优化问题的计算瓶颈的限制,只能处理低维系统。(2)我们的方法没有考虑乘性不确定性。


