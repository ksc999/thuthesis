% !TeX root = ../thuthesis-example.tex

% 中英文摘要和关键字

\begin{abstract}
  本文研究了具备加性不确定性和凸控制输入边界的仿射多项式系统中,鲁棒控制障碍函数的验证和综合问题。在上述系统条件下,控制障碍函数的验证和综合问题可表示为多级多项式优化问题。其中,验证问题包含三个级别的优化:系统不确定性、控制输入和系统状态;而综合问题对控制障碍函数的候选参数也进行了优化。研究表明,通过对系统不确定性和控制输入这两个优化级别使用KKT条件,验证问题可以简化为单级多项式优化问题,综合问题可以简化为最小-最大多项式优化问题。进一步地,我们可以使用多级半正定松弛求解上述两个问题:对于验证问题转化而来的单级多项式优化问题,我们应用Lasserre's Hierarchy这一矩松弛方法;对于综合问题转化来的最小-最大多项式优化问题,我们使用平方和优化方法获得优化问题未知值函数的不断紧逼的多项式下界,并再次调用Lasserre's Hierarchy以最大化下界。两种松弛方法都拥有渐进收敛到全局最优的理论保证。我们还对多控制障碍函数的验证理论进行了深入研究。实验方面,我们对受控Van der Pol振荡器进行了深入研究,并对系统具备和不具备加性不确定性这两种情况分别进行了讨论。

  % 关键词用“英文逗号”分隔,输出时会自动处理为正确的分隔符
  \thusetup{
    keywords = {控制障碍函数,多项式优化,半定松弛,最小-最大优化},
  }
\end{abstract}

\begin{abstract*}
  This paper presents a study on the verification and synthesis of robust control barrier functions for affine polynomial systems subject to additive uncertainty and convex control input bounds. We demonstrate that the verification and synthesis problem of the control barrier function can be represented as multilevel polynomial optimization problems, wherein the verification problem entails three levels of optimization, namely system uncertainty, control input, and system state, and the synthesis problem optimizes the parameters of the control barrier function. To address these problems, we propose the use of KKT conditions for the optimization levels of system uncertainty and control input. Specifically, we reduce the verification problem to a single-level polynomial optimization problem and the synthesis problem to a min-max polynomial optimization problem. Moreover, we adopt multi-level positive semi-definite relaxation to solve these problems. We utilize the Lasserre's Hierarchy moment relaxation method to tackle the transformed single-level polynomial optimization problem and the sum-of-squares optimization method to obtain an ever-tightening polynomial lower bound for the unknown value function of the optimization problem derived from the synthesis problem. Subsequently, we apply Lasserre's Hierarchy to maximize the lower bound. Both relaxation methods guarantee asymptotic convergence to the global optimum. We also conduct an in-depth study of the validation theory for multiple control barrier functions. Additionally, we perform a comprehensive analysis of the controlled Van der Pol oscillator in two scenarios: with and without additive uncertainty. The experimental results demonstrate the effectiveness of the proposed method.

  % Use comma as separator when inputting
  \thusetup{
    keywords* = {control barrier function, polynomial optimization, semidefinite relaxation, min-max optimization},
  }
\end{abstract*}
