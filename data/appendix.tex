% !TeX root = ../thuthesis-example.tex

\chapter{补充内容}

% \section{性质~\ref{prop:boundeddual}的证明}
% \label{app:proof:prop:boundeddual}

% \begin{proof}
%   (1)考虑$\mathbb{U}$是一个多面体,且其没有退化的极值点。观察可得,“$\max_{u \in \mathbb{U}} L_gb(x, \theta) u$”是一个关于$u$的线性函数(当$(x, \theta)$固定的时候)。由于$\mathbb{U}$没有退化的极值点,在最优解$u^\star$处,必有$m' \le m$个线性无关的约束被激活。因此,共有$m'$个非零的$\zeta_i^\star$~\cite{bertsimas97book-lp}。不失一般性地,我们假设$i = 1, \dots, m'$为被激活的约束。则KKT的stationary条件~\eqref{eq:kktstationary}可以被表示为:
%   \begin{eqnarray}
%     \underbrace{
%       \left[ \begin{matrix}
%         w_1 & \cdots & w_{m'}
%       \end{matrix} \right]
%     }_{:= W \in \mathbb{R}^{m \times m'}} = L_gb(x, \theta)
%   \end{eqnarray}
%   而这暗含了$(\zeta^\star)^T (W^TW) (\zeta^\star) = \parallel L_gb(x, \theta) \parallel^2$。当$W^TW \succ 0$时(亦即,$\left\{ w_i \right\}_{i=1}^{m'}$线性无关),我们有
%   \begin{eqnarray}
%     \left\lVert \zeta^\star \right\rVert \le \frac{
%       \left\lVert L_gb(x, \theta) \right\rVert^2
%     }{
%       \lambda_{\min} (W^T W)
%     }
%   \end{eqnarray}
%   最后,我们观察到$\left\lVert L_gb(x, \theta) \right\rVert^2$是有界的。这是因为它在紧集$\partial \mathcal{C} \times \Theta$上是光滑的。因此,$\left\lVert \zeta^\star \right\rVert^2$是有界的。

%   (2)考虑$\mathbb{U}$是一个箱型,它的每一个维度的取值范围都是$[-w_i, w_i]$。如此一来,KKT条件中的complementarity条件~\eqref{eq:kktcomp}说明了以下两者中有且仅有一者成立:第一,$\zeta_i^\star = 0$;第二,$\zeta_i^\star \ne 0$,但是$u_i^\star = \pm w_i$。在第二种情形下,使用KKT条件中的stationary条件~\eqref{eq:kktstationary},我们有
%   \begin{eqnarray}
%     \zeta_i^\star = \frac{
%       [L_gb(x, \theta)]_i
%     }{2u_i^\star} \Rightarrow 
%     (\zeta_i^\star)^2 = \frac{
%       [L_gb(x, \theta)]_i^2
%     }{4 w_i^2}, i = 1, \dots, m
%   \end{eqnarray}
%   其中$[L_gb(x, \theta)]_i$表示了$L_gb(x, \theta)$的第$i$项。由于$[L_gb(x, \theta)]_i$是有界的,$(\zeta_i^\star)^2$也是有界的。

%   (3)现考虑$\mathbb{U}$是一个椭球,且其被一个单独的约束$u^T W u \le 1$所定义。则KKT条件的complentarity条件~\eqref{eq:kktcomp}说明了以下两者中有且仅有一者成立:第一,$\zeta^\star=0$;第二,$\zeta^\star \ne 0$,但是$(u^\star)^T W (u^\star) = 1$。在第二种条件的情况下,使用KKT条件中的stationary条件~\eqref{eq:kktstationary},我们有
%   \begin{eqnarray}
%     (\zeta^\star)^2 = \frac{
%       \left\lVert L_gb(x, \theta) \right\rVert^2
%     }{4 \left\lVert W u^\star \right\rVert^2} 
%     \le \frac{
%       \left\lVert L_gb(x, \theta) \right\rVert^2
%     }{4 \lambda_{\min}(W)} 
%   \end{eqnarray}
%   由于$\left\lVert L_gb(x, \theta) \right\rVert^2$在$\partial \mathcal{C} \times \Theta$上是有界的,$(\zeta^\star)^2$在$\partial \mathcal{C} \times \Theta$上也是有界的。
% \end{proof}

% \section{对于系统~\eqref{eq:cleanvdpodynamics}和控制障碍函数~\eqref{eq:cleanvdpocbf},证明~\eqref{eq:standardpop}中的$y$有界}
% \label{app:bound:y:cleanvdp}

% 明显地,$\left\lVert x \right\rVert^2 \le \theta$,$u^2 \le u_{\max}^2$是有界的。然而,我们还需要证明$\zeta^\star$是有界的。KKT条件的complentarity条件~\eqref{eq:kktcomp}说明了以下两者中有且仅有一者成立:第一,$\zeta^\star=0$;第二,$\zeta^\star \ne 0$,但是$c_{u,i}(u) = 0$,这意味着$u^\star = \pm u_{\max}$。在第二种情况下,使用KKT条件中的stationary条件~\eqref{eq:kktstationary},我们有
% \begin{eqnarray}
%   2\zeta^\star u^\star = L_gb(x, \theta) = -2 x_1 x_2
% \end{eqnarray}
% 求解其中的$\zeta^\star$,我们有
% \begin{eqnarray}
%   \zeta = \frac{-x_1 x_2}{u^\star} \Rightarrow 
%   (\zeta^\star)^2 = \frac{x_1^2 x_2^2}{u_{\max}^2} \le \frac{
%     (x_1^2 + x_2^2)^2
%   }{4 u_{\max}^2} = \frac{\theta_2}{4 u_{\max}^2}
% \end{eqnarray}

% \section{性质~\ref{prop:wrongvdpocbf}的证明}
% \label{app:proofwrongvdpocbf}

% \begin{proof}
%   我们有
%   \begin{subequations}
%     \begin{eqnarray}
%       L_fb(x, \theta) =& -x_2^2 (1 - x_2^2) \\
%       L_gb(x, \theta) =& -2x_1 x_2 \\
%       L_Jb(x, \theta) =& -2x_2
%     \end{eqnarray}
%   \end{subequations}
%   作为结果,我们有:
%   \begin{subequations}
%     \begin{eqnarray}
%       V_u^\star =& \displaystyle \max_{u^2 \le u_{\max}^2} (-2 x_1 x_2) u 
%       = 2 u_{\max} |x_1 x_2| \\
%       V_\epsilon^\star =& \displaystyle \min_{
%         \left\lVert \epsilon \right\rVert \le M_\epsilon
%       } -2 x_2 \epsilon 
%       = -2 M_\epsilon |x_2|
%     \end{eqnarray}
%   \end{subequations}
%   以及
%   \begin{eqnarray}
%     V(\theta) = \min_{\left\lVert x \right\rVert^2 = \theta} 
%     -x_2^2 (1 - x_2^2) + u_{\max} |x_1 x_2| - M_\epsilon |x_2|
%   \end{eqnarray}
%   我们选择$x_2 = 0, x_1 = \pm \sqrt{\theta}$,可以得到$V(\theta) \le 0$。
% \end{proof}

% \section{对于系统~\eqref{eq:vdpodynamics}和控制障碍函数~\eqref{eq:ellipsoidalcbf},证明~\eqref{eq:standardpop}中的$y$有界}
% \label{app:bound:y:uncertainvdp}

% 显然,$u^2 \le u_{\max}$是有界的。而$x$也是有界的,这是因为
% \begin{eqnarray}
%   \max_{x^T A x = 1} \overset{v := A^{1/2}x}{=} 
%   \max_{v^T v = 1} v^T A^{-1} v = \frac{1}{\lambda_{\min}(A)}
% \end{eqnarray}
% 其中$\lambda_{\min}$和$\lambda_{\max}$表示了$A$的特征值中的最大值和最小值。注意到
% \begin{eqnarray}
%   \lambda_{\min} (\lambda_{\max} + \lambda_{\min}) \ge \lambda_{\min} \lambda_{\max} = \theta_1 \theta_2 - \theta_3^2 \Rightarrow \\
%   \lambda_{\min} \ge \frac{
%     \theta_1 \theta_2 - \theta_3^2
%   }{
%     \lambda_{\min} + \lambda_{\max}
%   } = \frac{
%     \theta_1 \theta_2 - \theta_3^2
%   }{
%     \theta_1 + \theta_2
%   }
% \end{eqnarray}
% 因此,$x$是有界的,因为
% \begin{eqnarray}
%   \left\lVert x \right\rVert^2 \le \frac{1}{\lambda_{\min}(A)} 
%   \le \frac{
%     \theta_1 + \theta_2
%   }{
%     \theta_1 \theta_2 - \theta_3^2
%   }
% \end{eqnarray}
% 现在考虑$z^2 = \left\lVert L_Jb(x, \theta) \right\rVert^2$:
% \begin{eqnarray}
%   \left\lVert L_Jb(x, \theta) \right\rVert^2 = 
%   \left\lVert -2 x^T A J(x) \right\rVert^2 = 
%   4 x^T A J(x)J(x)^T A x
% \end{eqnarray}
% 注意到$J(x)J(x)^T \preceq \textbf{I}$,这就意味着
% \begin{eqnarray}
%   \left\lVert L_Jb(x, \theta) \right\rVert^2 \le 
%   4x^T A^2 x \le 4 \lambda_{\max}(A) \le 4(\theta_1 + \theta_2)
% \end{eqnarray}
% 现在还需要证明$\zeta^\star$也是有界的。KKT条件能够告诉我们以下两者中有且仅有一种成立:第一,$\zeta^\star = 0$;第二,
% \begin{eqnarray}
%   \zeta^\star = \frac{L_gb(x, \theta)}{2u^\star}, \quad (u^\star)^2 = u_{\max}^2
% \end{eqnarray}
% 我们将$(L_gb(x, \theta))^2$写为
% \begin{eqnarray}
%   (L_gb(x, \theta))^2 = 
%   \left\lVert -2 x^T Ag \right\rVert^2 = 4 x^T Agg^T Ax
% \end{eqnarray}
% 其中$gg^T \preceq (x_1^2 + x_2^2) \textbf{I}$,我们可以得到
% \begin{eqnarray}
%   (L_gb(x, \theta))^2 \le 
%   4 \left\lVert x \right\rVert^2 x^T A^2 x \le 
%   4 \frac{
%     (\theta_1 + \theta_2)^2
%   }{
%     \theta_1 \theta_2 - \theta_3^2
%   } \\
%   \Rightarrow (\zeta^\star)^2 = 
%   \frac{
%     (L_gb(x, \theta))^2
%   }{4 u_{\max}^2} \le 
%   \frac{
%     (\theta_1 + \theta_2)^2
%   }{
%     u_{\max}^2 (\theta_1 \theta_2 - \theta_3^2)
%   }
% \end{eqnarray}

\section{计算矩}\label{sec:app:computemoments}

\subsection{无不确定性的受控Van der Pol振荡器}
在这种情况下,我们使用的是圆形控制障碍函数。考虑参数空间$\Theta \in [\theta_{\min}, \theta_{\max}]$,令$\Delta_\theta = \theta_{\max} - \theta_{\min}$成为$\Theta$的“长度”。我们计算\eqref{eq:computemoments}的矩
\begin{eqnarray}
  \gamma_\beta = \int_{\theta_{\min}}^{\theta_{\max}} \theta^\beta d \psi (\theta) = 
  \frac{1}{\Delta_\theta} \int_{\theta_{\min}}^{\theta_{\max}} \theta^\beta d\theta = 
  \frac{
    \theta_{\max}^{\beta+1} - \theta_{\min}^{\beta+1}
  }{(\beta + 1) \Delta_\theta}
\end{eqnarray}

\subsection{具备不确定性的受控Van der Pol振荡器}
这种情况下,我们使用椭圆形控制障碍函数。对于~\eqref{eq:ellipsoidalparam}所定义的$\Theta$,我们计算~\eqref{eq:computemoments}:
\begin{eqnarray}
  & \gamma_\beta = \displaystyle \int_{\Theta} \theta_1^{\beta_1} \theta_2^{\beta_2} \theta_3^{\beta_3} d \psi(\theta) \nonumber \\
  = & \displaystyle \int_{\theta_1} \int_{\theta_2} \int_{\theta_3} \theta_1^{\beta_1} \theta_2^{\beta_2} \theta_3^{\beta_3} \left(  
    \frac{1}{\text{vol}\Theta} d \theta_1 d \theta_2 d \theta_3
    \right) \nonumber \\
  = & \displaystyle \frac{1}{\text{vol}\Theta} \int_{\theta_1} \int_{\theta_2} \theta_1^{\beta_1} \theta_2^{\beta_2} \left(
    \int_{-\xi \sqrt(\theta_1 \theta_2)}^{\xi \sqrt(\theta_1 \theta_2)} \theta_3^{\beta_3} d \theta_3
    \right) \nonumber \\
  = & \displaystyle \frac{
    \xi^{\beta_3 + 1} (1 - (-1)^{\beta_3 + 1})
  }{
    (\beta_3 + 1) \text{vol}(\Theta)
  } \int_{\theta_1} \int_{\theta_2} 
  \theta_1^{\beta_1 + \frac{\beta_3}{2}} 
  \theta_2^{\beta_2 + \frac{\beta_3}{2}} d \theta_1 d \theta_2 \nonumber \\
  = & \displaystyle \frac{
    \xi^{\beta_3 + 1} (1 - (-1)^{\beta_3 + 1})
  }{
    (\beta_3 + 1) \text{vol}(\Theta)
  } \frac{
    \bar{\theta}^{\tilde{\beta}_1} - \underbar{\theta}^{\tilde{\beta}_1}
  }{\tilde{\beta}_1}
  \frac{
    \bar{\theta}^{\tilde{\beta}_2} - \underbar{\theta}^{\tilde{\beta}_2}
  }{\tilde{\beta}_2}
\end{eqnarray}
其中,$\text{\Theta}$是$\Theta$的体积(一个常数,我们不需要关心)。而$\tilde{\beta}_1 = \beta_1 + \frac{\beta_3}{2}$,$\tilde{\beta}_2 = \beta_2 + \frac{\beta_3}{2}$。


